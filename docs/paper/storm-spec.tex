\documentclass[11pt,a4paper]{article}
\usepackage[margin=1in]{geometry}
\usepackage{amsmath,amssymb,amsfonts}
\usepackage{array}
\usepackage{hyperref}
\usepackage{courier}

\hypersetup{
  colorlinks=true,
  linkcolor=blue,
  urlcolor=blue,
  citecolor=blue
}

\title{The Rainstorm Hash Function: Formal Specification (Draft)}
\author{Cris (DOSAYGO Corp.), 2023}
\date{\today}

\begin{document}
\maketitle

\section*{Overview}
Rainstorm is a family of keyed hash functions designed by Cris (DOSAYGO Corp.) in 2023. It produces fixed-size outputs of 64, 128, 256, or 512 bits from arbitrary-length byte inputs. Rainstorm uses a 1024-bit internal state and processes input data in 512-bit (64-byte) blocks. Its constants and rotation values were chosen based on extensive testing to ensure desirable mixing properties.

While Rainstorm was created with cryptographic aims, it is currently undergoing evaluation, and no formal proofs of security exist yet. The community is encouraged to analyze and test Rainstorm, applying techniques ranging from heuristic analysis to rigorous cryptography methods.

\section*{Domain and Range}
\textit{Input:} A message $M$ is a sequence of bytes $(M[0], \ldots, M[L-1])$ with arbitrary length $L \ge 0$.  
\textit{Seed:} A 64-bit unsigned integer $s$ (default $s=0$). Different seeds produce distinct hash variants.  
\textit{Output:} A digest of length $N \in \{64, 128, 256, 512\}$ bits.

\section*{Parameters and Constants}
Rainstorm uses fixed 64-bit constants and rotation amounts that were selected through computational searches and empirical testing to encourage strong avalanche properties.

Constants:
\[
P = 2^{64}-1 - 58, \quad Q = 13166748625691186689, \quad R = 1573836600196043749
\]
\[
S = 1478582680485693857, \quad T = 1584163446043636637, \quad U = 1358537349836140151
\]
\[
V = 2849285319520710901, \quad W = 2366157163652459183
\]

\[
K = [P, Q, R, S, T, U, V, W]
\]

Rotation constants:
\[
Z = [17, 19, 23, 29, 31, 37, 41, 53]
\]

Counter values:
\[
\text{CTR}_{\text{LEFT}} = 0xefcdab8967452301_{16}, \quad
\text{CTR}_{\text{RIGHT}} = 0x1032547698badcfe_{16}
\]

\section*{State Initialization}
For a given message length $L$, seed $s$, and output length $N$, the internal state $h$ is a vector of 16 64-bit words.

Define:
\[
c_0 = 1,\quad
c_1 = 2,\quad
c_2 = 2,\quad
c_3 = 3,\quad
c_4 = 5,\quad
c_5 = 7,\quad
c_6 = 11,\quad
c_7 = 13,
\]
\[
c_8 = 17,\quad
c_9 = 19,\quad
c_{10} = 23,\quad
c_{11} = 29,\quad
c_{12} = 31,\quad
c_{13} = 37,\quad
c_{14} = 41,\quad
c_{15} = 43.
\]

Then:
\[
h[i] = s + L + c_i \quad \text{for } i=0,\ldots,15.
\]

\section*{Block Processing and Endianness}
Rainstorm processes the input in 512-bit (64-byte) chunks. Each chunk is interpreted as eight 64-bit words in little-endian format:
\[
data[i] = \text{LE\_decode\_64}(B[8i : 8i+7]), \quad i=0,\ldots,7.
\]

The output is produced in little-endian format for each 64-bit word of $h$.

\section*{Rounds and Weak Function}
For each 64-byte chunk, Rainstorm performs 4 rounds of an internal function $\text{weakfunc}(h, data, left)$. The boolean $left$ is toggled each round.

Define $\text{ROTR}(x,n)$ as the 64-bit rotate-right operation on $x$ by $n$ bits.

\section*{The Weak Function}
\[
\text{weakfunc}(h, data, left)
\]

\textit{If $left = \text{true}$:}
\[
ctr = \text{CTR}_{\text{LEFT}}
\]
For $i=0,\ldots,7$:
\[
h[i] \leftarrow h[i] \oplus data[i], \quad
h[i] \leftarrow h[i] - K[i], \quad
h[i] \leftarrow \text{ROTR}(h[i], Z[i])
\]
\[
h[i+8] \leftarrow h[i+8] \oplus h[i], \quad
ctr \leftarrow ctr + h[i], \quad
h[(i+1)\bmod 8] \leftarrow h[(i+1)\bmod 8] - ctr
\]

\textit{If $left = \text{false}$:}
\[
ctr = \text{CTR}_{\text{RIGHT}}
\]
For $i=8,\ldots,15$:
\[
h[i] \leftarrow h[i] \oplus data[i-8], \quad
h[i] \leftarrow h[i] - K[i-8], \quad
h[i] \leftarrow \text{ROTR}(h[i], Z[i-8])
\]
\[
h[i-8] \leftarrow h[i-8] \oplus h[i], \quad
ctr \leftarrow ctr + h[i], \quad
h[((i-7)\bmod 8)+8] \leftarrow h[((i-7)\bmod 8)+8] - ctr
\]

\section*{Padding}
After processing all full 64-byte chunks, if any bytes remain, a final padded block is formed.

Let $r$ be the remaining length (0 to 63). Initialize a 64-byte array $temp$ with $(0x80 + r) \bmod 256$ in every byte, then overwrite the first $r$ bytes of $temp$ with the remaining data.

No length encoding beyond this padding is performed.

Process this final block with 4 rounds of $\text{weakfunc}$, then:
\[
\text{for } i=0 \text{ to } 7:\quad h[i] \leftarrow h[i] - h[i+8].
\]

If $N > 64$ bits, perform $\max(N/64, 2)$ additional calls of $\text{weakfunc}(h, temp, true)$ to further mix the state.

\section*{Output and Final Rounds}
After processing the final padded block and performing the subtraction step
\[
\text{for } i=0 \text{ to } 7:\quad h[i] \leftarrow h[i] - h[i+8],
\]
an additional finalization step is applied if $N > 64$ bits. Define:
\[
\text{FINAL\_ROUNDS} = 2.
\]

If the desired output length $N$ is greater than 64 bits, apply $\max(N/64, \text{FINAL\_ROUNDS})$ additional calls to
\[
\text{weakfunc}(h, temp, \text{true})
\]
using the same $temp$ block (the final padded block).

This ensures further mixing of the internal state before extraction of the final digest.

\subsection*{Extracting the Output}
The final hash is derived from the first $N/64$ 64-bit words of $h[0 \ldots 7]$ in little-endian format:

\begin{itemize}
  \item $N=64$ bits: output $h[0]$
  \item $N=128$ bits: output $(h[0], h[1])$
  \item $N=256$ bits: output $(h[0], h[1], h[2], h[3])$
  \item $N=512$ bits: output $(h[0], h[1], h[2], h[3], h[4], h[5], h[6], h[7])$
\end{itemize}

Each 64-bit word is written to the output buffer in little-endian byte order. No further transformations are performed.

\medskip

This finalization and extraction process ensures that for each supported output size, the resulting hash is fully defined by the preceding steps, including the additional final rounds for outputs greater than 64 bits.

\section*{Empirical Results and Analysis Invitation}
Rainstorm’s design choices were guided by empirical testing, including runs through \href{https://gitlab.com/fwojcik/smhasher3}{SMHasher3} (by Frank J. T. Wojcik). Tests indicate strong avalanche characteristics and minimal bias, suggesting that small changes in input bits produce large, unpredictable changes in output bits. Preliminary performance benchmarks show around 247 cycles per hash for small keys and bulk speeds of about 1.92 GiB/s at 3.5 GHz.

These observations are encouraging but are not conclusive cryptographic proofs. Rainstorm has not been formally analyzed by the broader cryptographic community. Interested researchers and practitioners, whether new to cryptanalysis or established in the field, are invited to scrutinize Rainstorm’s construction, test it against known cryptographic attacks, and share their findings.

\section*{Test Vectors}
Below are sample test vectors (with $s=0$ and $N=256$):

\begin{verbatim}
Message: ""
Digest: e3ea5f8885f7bb16468d08c578f0e7cc15febd31c27e323a79ef87c35756ce1e

Message: "abcdefghijklmnopqrstuvwxyzABCDEFGHIJKLMNOPQRSTUVWXYZ0123456789"
Digest: 9e07ce365903116b62ac3ac0a033167853853074313f443d5b372f0225eede50

Message: "The quick brown fox jumps over the lazy dog"
Digest: f88600f4b65211a95c6817d0840e0fc2d422883ddf310f29fa8d4cbfda962626

Message: "The quick brown fox jumps over the lazy cog"
Digest: ec05208dd1fbf47b9539a761af723612eaa810762ab7a77b715fcfb3bf44f04a

Message: "The quick brown fox jumps over the lazy dog."
Digest: 822578f80d46184a674a6069486b4594053490de8ddf343cc1706418e527bec8

Message: "After the rainstorm comes the rainbow."
Digest: 410427b981efa6ef884cd1f3d812c880bc7a37abc7450dd62803a4098f28d0f1

Message: "@@@@@@@@@@@@@@@@@@@@@@@@@@@@@@@@@@@@@@@@@@@@@@@@@@@@@@@@@@@@@@@@"
Digest: 47b5d8cb1df8d81ed23689936d2edaa7bd5c48f5bc463600a4d7a56342ac80b9
\end{verbatim}

\section*{Conclusion}
This document provides a formal specification of Rainstorm’s design, constants, and processing steps, as well as initial empirical observations. The cryptographic community is invited to analyze Rainstorm for properties such as collision resistance, preimage resistance, and differential security. Only through public scrutiny and rigorous testing can Rainstorm be evaluated as a potential candidate for widespread cryptographic use.

\end{document}

